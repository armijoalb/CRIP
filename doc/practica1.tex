\input{preambuloSimple.tex}
\usepackage{enumerate}
\usepackage{listings}
\usepackage{amsmath}%
\usepackage{amsfonts}%
\usepackage{amssymb}%
\usepackage{MnSymbol}%
\usepackage{wasysym}%
\usepackage{graphicx}
\usepackage{grffile}
\usepackage[spanish]{babel}
\usepackage{babel}
\usepackage{tikz}
\usetikzlibrary{babel}
\usepackage{color}
\usepackage{hyperref}
\usepackage{setspace}

\definecolor{dkgreen}{rgb}{0,0.6,0}
\definecolor{gray}{rgb}{0.5,0.5,0.5}
\definecolor{mauve}{rgb}{0.58,0,0.82}
\definecolor{LightCyan}{rgb}{0.88,1,1}

\lstset{frame=tb,
	language=Java,
	aboveskip=3mm,
	belowskip=3mm,
	showstringspaces=false,
	columns=flexible,
	basicstyle={\small\ttfamily},
	numbers=none,
	numberstyle=\tiny\color{gray},
	keywordstyle=\color{blue},
	commentstyle=\color{dkgreen},
	stringstyle=\color{mauve},
	breaklines=true,
	breakatwhitespace=true,
	tabsize=3
}

%----------------------------------------------------------------------------------------
%	TÍTULO Y DATOS DEL ALUMNO
%----------------------------------------------------------------------------------------

\title{
\normalfont \normalsize
\textsc{ Programación de Dispositivos Móviles (2017-2018)\\
			Grado en Ingeniería Informática\\ 
			Universidad de Granada}
			\\ [25pt] % Your university, school and/or department name(s)
\horrule{0.5pt} \\[0.4cm] % Thin top horizontal rule
\huge Tutorial 1\\ % The assignment title
\horrule{2pt} \\[0.5cm] % Thick bottom horizontal rule
}

\author{ Juan Alberto Martínez López \\ Alberto Armijo Ruiz \\}

\date{\normalsize\today} % Incluye la fecha actual



%----------------------------------------------------------------------------------------
% DOCUMENTO
%----------------------------------------------------------------------------------------

\begin{document}

\maketitle % Muestra el Título

\includegraphics[width=1\linewidth]{ugr}

\newpage %inserta un salto de página

\tableofcontents % para generar el índice de contenidos

%\listoffigures

%\listoftables

\newpage

%----------------------------------------------------------------------------------------
%	Introducción
%----------------------------------------------------------------------------------------

\section*{Introducción}

Para esta práctica hemos desarrollado el algoritmo Miller-Rabin para decidir si un número es posible primo o no es primo, para ello hemos desarrollado dos versiones del algoritmo: Una en la que se realizan n aplicaciones para comprobar la primalidad y otro con una lista de números naturales con los que se comprueba el test para el primo y cada una de las bases de la lista. Para calcular la primalidad utilizamos el algoritmo de logaritmo discreto en el que comprobamos la existencia dado a,b y p de $\log_a (b) \mod n$ y si se cumple el número es primo.

\section{Evaluación de tiempos}

Para el algoritmo Miller-Rabin con 2000 aplicaciones podemos apreciar muy poco aumento del tiempo (Tabla 1.1), por lo que podemos concluir que el procesamiento de la primalidad no tiene una carga grande y no hay problema a la hora de calcular números grandes.
\begin{table}[htbp]
	\begin{center}
		\begin{tabular}{|l|l|}
			\hline
			\rowcolor{LightCyan}
			Número Primo & Tiempo ejecución (seg) \\ \hline
			57347&  0.008280 \\ \hline 
			468577&  0.009140 \\ \hline 
			5555567&  0.013386 \\ \hline 
			87654337&  0.012030 \\ \hline 
			987654323&  0.014271 \\ \hline 
			3141592661&  0.019104 \\ \hline 
			11111111113&  0.020818 \\ \hline 
			121212121223&  0.021233 \\ \hline 
			2718281828489&  0.025449 \\ \hline 
			16180339892149&  0.027566 \\ \hline 
			800000000000017&  0.031171 \\ \hline 
		\end{tabular}
		\caption{Tabla tiempos de ejecución para el algoritmo Miller Rabin}
		\label{tabla:millerrabin}
	\end{center}
\end{table}

\begin{figure}[H]
	\begin{center}
		\includegraphics[width=1\linewidth]{chart2}
		\caption{Tiempos vs Longitud clave Miller-Rabin}
		\label{figura: tiempos3}
	\end{center}
\end{figure}

Para el algoritmo del logaritmo discreto se han calculado una a y c aleatoria y hemos calculado la b a partir de $b = a^c\mod p$. En las tablas 1.2 y 1.3 si que se puede apreciar un aumento considerable de los tiempos ya que de un tiempo a otro podemos ver como se cuadruplican de un resultado a otro. \\

\begin{table}[htbp]
	\begin{center}
		\begin{tabular}{|l|l|}
			\hline
			\rowcolor{LightCyan}
			Longitud clave & Tiempo ejecución (seg) \\ \hline
			5 & 0.001187 \\ \hline 
			6 & 0.004052 \\ \hline
			7 & 0.018117 \\ \hline
			8 & 0.080128\\ \hline
			9 & 0.289543\\ \hline
			10 & 0.73591\\ \hline
			11 & 1.506318\\ \hline
			12 & 5.264323\\ \hline
			13 & 29.752512\\ \hline
			14 & 76.744032 \\ \hline
			15 & 612.546521\\ \hline
		\end{tabular}
		\caption{Tabla tiempos}
		\label{tabla:resumen}
	\end{center}
\end{table}


\begin{table}[htbp]
	\begin{center}
		\begin{tabular}{|l|l|l|l|l|}
			\hline 
			\rowcolor{LightCyan}
			A & B & P & Solución & Tiempos (s) \\ \hline
			6 & 50628 & 57347& 7 & 0.001187 \\ \hline 
			8 & 449605 & 468577& 11 & 0.004052 \\ \hline
			207 & 4374842 & 5555567& 104 & 0.018117 \\ \hline
			4007 & 8515459 & 87654337& 430 & 0.080128\\ \hline
			40756 & 118205788 & 987654323& 10748 & 0.289543\\ \hline
			20544 & 253647140 & 3141592661& 113 & 0.735911\\ \hline
			112354 & 9048018943 & 11111111113& 5658 & 1.506318\\ \hline
			1245628 & 49579028347 & 121212121223& 568985 & 5.264323\\ \hline
			87569 & 1342094524016 & 2718281828489& 5749833 & 29.752512\\ \hline
			568236 & 14717101287551 & 16180339892149& 389567512 & 76.744032\\ \hline
			4555786 & 778596955901441 & 800000000000017& 785951 & 612.546521\\ \hline
		\end{tabular}
		\caption{Tabla análisis de logaritmo.}
		\label{tabla:compleja}
	\end{center}
\end{table}

En la siguiente figura podemos apreciar como el crecimiento del tiempo de ejecución tiene forma de una función exponencial.\\

\begin{figure}[H]
	\begin{center}
		\includegraphics[width=1\linewidth]{chart}
		\caption{Tiempos vs Longitud clave}
		\label{figura: tiempos}
	\end{center}
\end{figure}

Si le metemos una escala logarítmica vemos como la función es casi lineal.
\begin{figure}[H]
	\begin{center}
		\includegraphics[width=1\linewidth]{chart1}
		\caption{Tiempos vs Longitud clave escala logaritmica}
		\label{figura: tiempos2}
	\end{center}
\end{figure}



Hemos calculado la media armónica entre un tiempo y el siguiente es de 3.47 el cual podemos aproximarlo a 4. De manera que dado un tamaño de X cifras, el tiempo aproximado de ejecución sería de $4^X$ por lo que para números grandes el tiempo necesario de ejecución sería enorme.\\

El algoritmo además ocupa una gran cantidad de RAM, ya que para la clave de mayor tamaño ha llegado a ocupar 4.2 Gb.\\

\section*{Conclusión}

Este algoritmo no tiene una gran eficiencia a la hora de procesar números grandes ya que si por ejemplo ejecutamos el algoritmo con una clave de 50 cifras tendríamos que $4^(50) = 4.02 e+19$ años, lo que sería demasiado tiempo para descodificar una clave de ese tamaño.\\

%\includegraphics[width=1\linewidth]{Capturas VRTK/prueba 43.1}\\

%\includegraphics[width=1\linewidth]{Capturas VRTK/prueba 43.2}\\

%\includegraphics[width=1\linewidth]{Capturas VRTK/prueba 43.3}


%----------------------------------------------------------------------------------------
%	REFERENCIAS
%----------------------------------------------------------------------------------------

%\section{Incluyendo las referencias}

%A lo largo del texto, tendrá que incluir referencias, para ello puede usar las
%notas a pie de página con \textbackslash footnote \footnote{Este es un ejemplo}
%o añadiendo citas con \textbackslash cite \cite{mrx05,prueba2}.
%\LaTeX puede gestionar las referencias mediante BibTex, que genera entradas bbl
%a partir de un archivo .bib. También puede incluirlas directamente en el documento
%fuente como entradas dentro de una sección (thebibliography) dentro del mismo
%documento (son las generadas en el archivo .bbl por BibTex). No obstante,
%se recomienda el uso del archivo .bib como buena práctica a seguir cuando se
%trabaje con \LaTeX.

%Para incluir las referencias usaremos la etiqueta \textbackslash bibliography
%especificando su estilo: \textbackslash bibliographystyle que, en nuestro caso,
%es plain.

%------------------------------------------------
%\newpage

%\bibliographystyle{plain} % hay varias formas de citar

%\begin{thebibliography}{99} %Principio Bibliografía (99 max elementos)

	%\bibitem{virtual} Virtualización. (24 de septiembre de 2015):
	%\url{http://www.vmware.com/latam/virtualization/how-it-works}.

%	\bibitem{proyectivo} \url{http://www.rac.es/ficheros/doc/00894.pdf} 
%	\bibitem{consP} \url{https://eva.fing.edu.uy/pluginfile.php/63526/mod_resource/content/2/Teorico/clase04_TransformacionesGeo.pdf}
%	\bibitem{homomorfismo} \url{https://en.wikipedia.org/wiki/Homography#Homographies_of_a_projective_line}
%	\bibitem{hart} Multiple View Geometry in Computer Vision (Second Edition) \textit{Richard Hartley, Andrew Zisserman}
%	\bibitem{szel} Computer vision: Algorithms and applications \textit{Richard Szeliski}	

%\end{thebibliography}


%\bibliography{citas} %archivo citas.bib que contiene las entradas




\end{document}